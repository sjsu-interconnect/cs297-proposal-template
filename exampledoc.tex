% Title of your project
\title{JestScript - A Whimsical Dive into Humorous Programming}

% Student's name
\def\myname{John von Neumann}

% Advisor's information
\def\adv{Dr. Donald Knuth}
\def\advemail{sample@sjsu.edu}

% Committee Member's Information 1
\def\comone{Dr. Edsger W. Dijkstra}
\def\comoneemail{sample@sjsu.edu}

% Committee Member's Information 2
\def\comtwo{Dr. Claude Shannon}
\def\comtwoemail{sample@sjsu.edu}
% Affiliation (if not SJSU CS regular faculty); Leave empty if SJSU CS regular faculty)
\def\comtwoaffil{}

% CS 298/299 Semester and Year
\def\semester{Summer}
\def\year{1822}


% Five to seven technical words that best describe your project (Keywords you use to search related works similar to your project on Google scholar)
\def\keywords{programming language, enjoyable coding, programming language design, user experience, light-weight programming language}


\maketitle % Do not remove


% -----------
\section{Introduction}
% Introduction to the field
In the realm of code seriousness, JestScript emerges as a playful endeavor, infusing the traditionally stoic world of programming with humor. 
% Introduction to the specific area of the study
Motivated by the desire to bring joy to coding, JestScript presents a unique take on syntax, error messages, and debugging experiences.

\makekeywords % Do not remove

% -----------
\section{Problem Definition and Motivation}
% What is a problem you are solving in your project? Why is the problem important to the research community or the public?
The motivation behind JestScript lies in challenging the stereotypical perception of coding as a dry and serious task. With a plethora of programming languages prioritizing functionality over fun, JestScript aims to provide developers with a light-hearted alternative, fostering a more enjoyable coding
atmosphere.

% -----------
\section{Results Achieved in CS 297}
% List a few concrete items that you have implemented, written, or created in CS 297
\begin{itemize}
  \item JestScript Compiler: The JestScript compiler that translates JestScript code into executable instructions for the target platform
  \item JestScript Syntax Guide: A detailed guide documenting the JestScript syntax, language rules, and conventions for developers.
\end{itemize}


% -----------
\section{Expected Deliverables in CS 298/299}
% List a few concrete items that you plan to implement, write, or create by the end of CS 298/299
\begin{itemize}
  \item Open-Source Repository: The public repository on a platform like GitHub containing the JestScript source code, documentation, and issue tracker.
  \item User Feedback Analysis Report: A report summarizing user feedback from testing phases, outlining areas of improvement and potential future enhancements.
\end{itemize}


% -----------
\section{Timeline and Milestones for CS 298/299}
% Define several milestones for your CS 298/299
% Decide a schedule to complete the milestones

\begin{figure}[H]
  \centering
  \begin{ganttchart}[y unit title=0.5cm,
    y unit chart=0.5cm,
    vgrid,hgrid, 
    title label anchor/.style={below=-1.6ex},
    title left shift=.05,
    title right shift=-.05,
    title height=1,
    progress label text={},
    bar height=0.7,
    group right shift=0,
    group top shift=.6,
    group height=.3]{1}{20}
  %labels
  \gantttitle{January}{4} 
  \gantttitle{February}{4} 
  \gantttitle{March}{4} 
  \gantttitle{April}{4} 
  \gantttitle{May}{4} \\
  %tasks
  \ganttbar{Compiler Tests}{1}{4} \\
  \ganttbar{Open Source Documentation}{3}{8} \\
  \ganttbar{User Experience Survey}{9}{11} \\
  \ganttbar{Survey Analysis}{12}{13} \\
  \ganttbar{Report Writing}{14}{17} \\
  \ganttbar{Defense Preparation}{18}{19}
  %relations 
  % \ganttlink{elem0}{elem1} 
  \end{ganttchart}
\end{figure}


% -----------
\nocite{*} % if you add entries to ref.bib, all items will be shown.
\bibliographystyle{ieeetr}
{\small \bibliography{ref-example}}